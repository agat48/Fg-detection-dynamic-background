\chapter{Analiza dostępnych rozwiązań}
\label{cha:analiza}

W rozdziale tym omówione zostaną wybrane dotychczasowo opracowane metody.

%---------------------------------------------------------------------------

\section{Flux Tensor with Split Gaussian Models}
\label{sec:FTSG}

Jednym z zaproponowanych rozwiązań omawianego w tej pracy problemu jest fuzja dwóch metod detekcji zmian na sekwencji wideo \cite{6910016}. Polega ona na wyliczeniu tensora przepływu (ang. \textit{Flux Tensor}) i modelowaniu tła za pomocą algorytmu bazującego na GMM opisanego w sekcji \ref{sec:GMM} ze zmienną liczbą modeli, dostosowującą się automatycznie przestrzennie i czasowo.
Metoda ta dzieli się na trzy główne moduły:
\begin{itemize}
\item
detekcja zmian na poziomie piksela - obliczane są osobno modele dla ruchu (\textit{flux tensor} - FT) i dla zmian jasności/kolorów na obrazie (\textit{split Gaussian model} - SG)
\item
fuzja otrzymanych wyników - stosując odpowiednie reguły łączone są modele FT i SG, aby zredukować ilość fałszywych detekcji
\item
klasyfikacja obiektów - obsługa przypadków obiektów zatrzymanych bądź usuniętych ze sceny
\subsection{Flux Tensor}
Tensor przepływu pozwala na wykrycie ruchu pomiędzy kolejnymi ramkami nagrania. 
\end{itemize}
\section{Weightless Neutral Networks}
Praca \cite{6910014}
\section{Self-tuning Background Subtraction Algorithm}
\label{sec:BinWang}
Choć w ostatnich latach możliwości obliczeniowe maszyn, zarówno pod względem ilości przetwarzanych danych, jak i prędkości, rozwijały się bardzo dynamicznie, obecni inżynierowie borykają się z problemami nie do przezwyciężenia. Częstotliwość taktowania procesorów o dotychczas stosowanej architekturze nie może być już bardziej zwiększana - spowodowane jest to fizycznymi ograniczeniami sprzętu. Sekwencyjne wykonywanie poleceń uzależnia więc czas działania algorytmu od prędkości obliczeniowej jednostki wykonującej operacje. Dla problemu opisywanego w tej pracy szybkość przetwarzania jest szczególnie ważna - standardowy format wideo to 25 klatek na sekundę. Oznacza to, iż w celu detekcji w czasie rzeczywistym, jedna ramka o określonej rozdzielczości musi zostać przetworzona w czasie 1/25 sekundy. Dlatego też autorzy rozwiązania opisanego w artykule \cite{6910012} postanowili zmierzyć się z tym problemem, opracowując metodę, która wykonuje potrzebne obliczenia dla każdego piksela niezależnie, dokonując klasyfikacji jedynie na podstawie historii jego jasności. Takie podejście umożliwia implementację w architekturze równoległej, jak choćby z wykorzystaniem mocy obliczeniowej GPU (ang. \textit{Graphics processing unit} - procesor graficzny) czy układów FPGA (ang. \textit{Field-programmable gate array}).

\section{SuBSENSE}
\cite{stflexible}
\section{Spectral-360}
W pracy \cite{6910013}



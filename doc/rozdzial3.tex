\chapter{Analiza rozwiązań}
\label{cha:analiza}

W rozdziale tym omówione zostaną wybrane dotychczasowo opracowane metody.

%---------------------------------------------------------------------------

\section{Flux Tensor with Split Gaussian Models}
\label{sec:FTSG}

Jednym z zaproponowanych rozwiązań omawianego w tej pracy problemu jest fuzja dwóch metod detekcji zmian na sekwencji wideo \cite{6910016}. Polega ona na wyliczeniu tensora przepływu (ang. \textit{Flux Tensor}) i modelowaniu tła za pomocą algorytmu bazującego na MoG (ang. \textit{Mixture of Gaussians}) ze zmienną liczbą modeli, dostosowującą się automatycznie przestrzennie i czasowo. \textcolor{red}{UZUPEŁNIJ O MoG!!!}
Metoda ta dzieli się na trzy główne moduły:
\begin{itemize}
\item
detekcja zmian na poziomie piksela - obliczane są osobno modele dla ruchu (\textit{flux tensor} - FT) i dla zmian jasności/kolorów na obrazie (\textit{split Gaussian model} - SG)
\item
fuzja otrzymanych wyników - stosując odpowiednie reguły łączone są modele FT i SG, aby zredukować ilość fałszywych detekcji
\item
klasyfikacja obiektów - obsługa przypadków obiektów zatrzymanych bądź usuniętych ze sceny
\subsection{Flux Tensor}
Tensor przepływu pozwala na wykrycie ruchu pomiędzy kolejnymi ramkami nagrania. 
\end{itemize}
\section{Weightless Neutral Networks}
Praca \cite{6910014}
\section{Self-tuning Background Subtraction Algorithm}
Informacje o rozwiązaniu można znaleźć w artykule \cite{6910012}
\section{SuBSENSE}
\cite{stflexible}
\section{Spectral-360}
W pracy \cite{6910013}



\chapter{Wprowadzenie do tematyki pracy}
\label{cha:tematykaPracy}
\section{Segmentacja obiektów}
(cośtam) często interesujące są jedynie obiekty znajdujące się na pierwszym planie. Tło bowiem nie niesie żadnej istotnej informacji.
\section{Detekcja ruchu na scenie}
Detekcja ruchu, czyli \textit{Motion Detection}, to sposób wykrywania przemieszczania się obiektów na scenie względem ich sąsiedztwa. Polega ona na analizie kolejnych ramek z sekwencji wideo i badaniu zmian następujących pomiędzy nimi.\\
W wykrywaniu zmian pomagają algorytmy generacji tła. 
\section{Biblioteka OpenCV}
\begin{figure}
\centering

\includegraphics[width=82px]{img/ocv_logo}
\caption{Logo biblioteki OpenCV \cite{OpenCVLogo}}
%\floatfoot{Source: }
\end{figure}
OpenCV (ang. \textit{Open Source Computer Vision}) to dostępna od 2000 roku biblioteka, zawierająca implementacje najczęściej wykorzystywanych algorytmów wizyjnych. Jej główne zalety to dostępność na zasadach \textit{open source}, a także wieloplatformowość.\\
Została ona napisana w języku C, jednak istnieją specjalne nakłądki, pozwalające korzystać z niej także m. in. w C++, C\#, Python i języku Java. Udostępniony publicznie jest nie tylko kod źródłowy, ale także i narzędzia, pozwalające na samodzielną kompilację biblioteki, co pozwala na używanie jej w wielu środowiskach programistycznych.

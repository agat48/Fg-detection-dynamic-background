\chapter{Testy rozwiązań}
\label{cha:testy}
\renewcommand{\tablename}{Tabela}

W celu ewaluacji działania opisanych w rozdziale \ref{cha:analiza} metod detekcji przeprowadzone zostały testy każdego z rozwiązań. Miały one na celu zbadać nie tylko dokładność detekcji, ale także i szybkość wykonywania algorytmu. Jest ona szczególnie ważna ze względu na fakt konieczności przetwarzania danych z kamery wizyjnej w czasie rzeczywistym oraz możliwość zaprogramowania rozwiązania na urządzeniach wbudowanych. \\
Do przeprowadzenia testów użyte zostały sekwencje ze strony changedetection.net, dotyczące dynamicznego ruchu na scenie. Zmierzona została szybkość detekcji, wyrażana jako ilość przetworzonych ramek na sekundę (ang. \textit{fps - frames per second}). W oparciu o ręcznie wyznaczane dla każdej ramki modele referencyjne (\textit{groundtruth}) przeprowadzona została także analiza skuteczności rozpatrywanych rozwiązań. Polegała ona na wyznaczeniu wartości:
\begin{itemize}
\item TP - \textit{True Positives}, czyli ilości pikseli zaklasyfikowanych poprawnie jako pierwszy plan
\item FP - \textit{False Positives} - liczby pikseli wykrytych jako pierwszy plan, jednak w modelu referencyjnym będące tłem
\item TN - \textit{True Negatives} - ilości zgodnych z modelem klasyfikacji pikseli tła
\item FN - \textit{False Negatives}, oznaczającego liczbę pikseli z obszarów, na których nie został wykryty ruch pomimo przemieszczenia obiektów sceny
\end{itemize}
Wartość $F$ wyznaczana na podstawie wzorów:
\begin{align}
\begin{split}
P &= \frac{TP}{TP+FP} \\
R &= \frac{TP}{TP+FN} \\
F &= \frac{2PR}{P+R}
\end{split}
\end{align}
jest wskaźnikiem stopnia poprawności detekcji.
\\
\begin{LARGE}
\textcolor{red}{WYNIKI DLA FTSG, CwisarDH itd.}
\end{LARGE}

\section{Wyniki}
\begin{table}[t]
\caption{Porównanie badanych metod}
\label{tab:results}
\centering
\begin{tabular}{|l*{6}{|c}|}
  \hline 
  \textbf{Metoda} & \textbf{fps} & \textbf{1} & \textbf{2} & \textbf{1} & \textbf{2} & \textbf{5}\\
  \hline
  Flux Tensor with Split Gaussian Models & 0.5 & 0 & 0 & 0 & 0 & 0\\
  \hline
  CwisarDH & 0.5 & 0 & 0 & 0 & 0 & 0\\
  \hline
  Self-turning Background Subtraction & 0.5 & 0 & 0 & 0 & 0 & 0\\
  \hline
  SubSENSE & 0.5 & 0 & 0 & 0 & 0 & 0\\
  \hline
  Średnia bieżąca & 0.5 & 0 & 0 & 0 & 0 & 0\\
  \hline
  Gaussian Mixture Models & 0.5 & 0 & 0 & 0 & 0 & 0\\
  \hline
  ViBE & 0.5 & 0 & 0 & 0 & 0 & 0\\
  \hline
  PBAS & 0.5 & 0 & 0 & 0 & 0 & 0\\
  \hline
\end{tabular}
\end{table}
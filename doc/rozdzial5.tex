\chapter{Wnioski}
\label{sec:wnioski}
Dynamiczne tło na scenie znacznie utrudnia segmentację obiektów pierwszoplanowych. Przetestowane rozwiązania dają jednak możliwość radzenia sobie z tym problemem. Pamiętać należy jednak, iż w przetwarzaniu w czasie rzeczywistym istotny jest czas wykonywania algorytmów. Nie bez znaczenia staje się więc fakt, iż niektóre z rozwiązań przedstawionych w tej pracy można zrównoleglić, a więc napisać program w taki sposób, by potrzebne obliczenia wykonywały się niezależnie od siebie z wykorzystaniem mocy obliczeniowej dostępnych w danej technologii narzędzi, czy to fizycznych (FPGA), czy wirtualnych - wielowątkowość w C++. Choć więc rozwiązanie przedstawione w \ref{sec:FTSG} zdaje się dawać najdokładniejsze wyniki, do przemyślenia pozostaje kwestia, czy ważniejsza jest dokładność detekcji, czy szybkość wykonywania operacji.
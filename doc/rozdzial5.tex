\chapter{Wnioski}
\label{sec:wnioski}
Dynamiczne tło na scenie znacznie utrudnia segmentację obiektów pierwszoplanowych. Przetestowane rozwiązania dają jednak możliwość radzenia sobie z tym problemem. Na podstawie danych z tabeli \ref{tab:resultsAll} można wnioskować, iż najlepszą metodą segmentacji niezależnie od warunków sceny jest Flux Tensor with Split Gaussian Models. Pamiętać należy jednak, iż w przetwarzaniu w czasie rzeczywistym istotny jest czas wykonywania segmentacji pojedynczej ramki. Tabele \ref{tab:resultsHighway}-\ref{tab:resultsOverpass} pokazują, iż wspomniany algorytm działa najwolniej - niekiedy ponad trzykrotnie dłużej, niż inne testowane rozwiązania.
\paragraph{}
Segmentacja w przypadku pozostałych metod odbywa się na poziomie pojedynczego piksela. Oznacza to, iż możliwa jest ich implementacja w architekturze równoległej. Choć implementacje stworzone na potrzeby tej pracy wykonują obliczenia sekwencyjnie, dane ze strony changedetection.net oraz zawarte w artykułach dotyczących omawianych metod pokazują wyraźnie, iż wydajność algorytmów może zostać wielokrotnie poprawiona właśnie poprzez wprowadzenie podziału na sekcje przetwarzane równolegle. Dla przykładu, przeprowadzone na stronie changedetection.net testy rozwiązania omawianego w sekcji \ref{sec:BinWang} (Self-turning...) z wykorzystaniem procesora graficznego GPU i środowiska CUDA dały wyniki średnio 843 ramek na sekundę. Nie tylko więc metody te działają szybciej niż FTSG, ale dają również duże możliwości w dziedzinie implementacji sprzętowej.
\paragraph{}
Wybór najlepszego rozwiązania pozostaje więc kwestią otwartą. Nie da się bowiem jednoznacznie stwierdzić, czy ważniejsza jest dokładność, czy szybkość detekcji. 